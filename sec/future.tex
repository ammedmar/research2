
\section{Future work} \label{s:future}

\subsection{Steenrod operations and Khovanov homology}

As mentioned before, at odd primes Steenrod operations, originally defined nonconstructively, were given a cochain level description in \cite{medina2020maysteenrod}.
With F. Cantero-Mor\'{a}n (\textit{UAM}) we will work on developing a description of Steenrod operations at the simplicial cochain level that is dual to this one.
This work constitutes a generalization of the description of cup-$i$ products introduced in \cite{medina2021newformulas} which lead to advances in applied topology discussed in the previous section.
With these new formulas we will develop algorithms for the fast computation of Steenrod operations suitable for their incorporation into persistent homology.

Another area impacted by this work is the study of knots and links.
A wide range of invariants have been developed to distinguish knots and links with deep connections to physics and other areas of mathematics.
For example, V. Jones and E. Witten were awarded the Fields Medal in 1990 in part for their study of the Jones polynomial, an invariant whose ``categorification'' is known as Khovanov homology.

The dual description of cup-$i$ products introduced in \cite{medina2021newformulas} allowed Cantero-Mor\'{a}n to increase the discriminatory power of this invariant by defining Steenrod squares on mod 2 Khovanov homology \cite{cantero-moran2020khovanov}.
We will work on implementing these formulas into high-performance routines that could be used to distinguish knots in practice.
Additionally, our work dualizing the formulas for Steenrod cup-$(p,i)$ products will allow us to define Steenrod operations on mod $p$ Khovanov homology, as well as implementing effective methods for their computation.

This project will involved several graduate students and other young researcher, for which some financial support has already been secured.

\subsection{Secondary operations and Postnikov towers}

Steenrod operations can be understood as structure on cohomology that arises from lifting to the cochain level the commutativity relation satisfied by the cup product.
This is part of a broader conceptual pattern where secondary homological structure can be derived from relations bounding primary structure.
Steenrod operations, which we now regard as primary cohomological structure, satisfy Cartan and Adem relations.
With G. Brumfiel (\textit{Stanford}) and J.~Morgan (\textit{Columbia}) we produced cochains enforcing these relations in the case $p=2$, and will extend this work to all primes.
This will allow us to provide cochain models approximating spaces presented by certain Postnikov towers, as they have done for the Spin bordism spectrum.
Work motivated by the study of lattice field theories as surveyed in \cref{ss:physics}.

\subsection{Rational homotopy theory}

So far we have focused on the prime $p$ part of the homotopy theory of spaces.
There is also a rich theory started by D. Quillen and D. Sullivan, my former Ph.D. advisor, modeling algebraically the rational part.
One of the formulations of this algebraization is provided by $C_\infty$-coalgebras.
These are special kind of $E_\infty$-coalgebras where the cocommutativity holds strictly while coassociativity remains relaxed up to coherent homotopies.
I will construct a map from the $C_\infty$-operad controlling this structures to my model of the $E_\infty$-operad \cite{medina2020prop1} considered over the rationals.
Since, due to its finite presentation, my model can be used to describe very explicitly $E_\infty$-structures, we will use it to advance effective computational techniques in rational homotopy theory.
For example, in \cite{lawrence2014interval} R. Lawrence and D. Sullivan define an explicit $C_\infty$-coalgebra structure on the chains of the interval.
In \cite{buijs2020liemodels} this was extended nonconstructively to the chains on all simplices.
We expect to provide explicit descriptions of these and other fundamental $C_\infty$-coalgebras using these and other techniques.
These opens the possibility of using Massey products as finer invariants in applied topology, analogously to how Steenrod squares have been incorporated to it by my work.